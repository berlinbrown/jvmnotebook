% %-------------------------------------- % Scala and Lift
% %--------------------------------------
\chapter{More about the Java Virtual Machine}

One of the exciting trends to recently emerge from the Java community is the
concept of the JVM language. These technologies are all that you would expect
them to be. They are implementations of languages that run on the Java Virtual
Machine. Some are newly created and some are based on existing, more mature
languages. JRuby, Jython are two JVM languages based on CRuby and CPython.
Groovy, Scala, Clojure are three completely new JVM languages that were created
to add new language features that weren't supported by the core Java language.
Some must or can be compiled. Some run without compilation. You can easily
compile Scala code to Java bytecode. Clojure also allows this feature (ahead of
time compilation). Clojure and JRuby code can also run without having be
explicitly compiled. You can interact with the language. In most cases and with
most JVM languages, you have full access to existing libraries that were written
in pure Java. And normally you can access existing JVM language code from Java
(known as Java interoperability). In most cases, it is easier to access Java
calls from the JVM language than it is to call the language code from Java. It
really depends on the language. In the snippet below, there is a call to the
System static method, 'nanoTime' . Simply invoke the system like you would from
pure Java.

  For the more popular JVM languages, like JRuby and Jython, there isn't much of
  a difference between running code in their respective C implementations. JRuby
  is especially well known for being very portable. With JRuby release 1.3.1,
  JRuby is compatible with CRuby 1.8.6. Jython 2.5.0 was released last month and
  brings the level of compatibility to CPython versions 2.5. Django and other
  popular CPython based frameworks are able to work with Jython. You may be
  wondering, if the Java Virtual Machine language is compatible with the C
  language, native implementation, is there a loss in performance when running
  on the Java Virtual Machine? Is there a major loss in performance? That is
  this purpose of this document, how much time does it take for a particular
  piece of code to run in JVM language? How long does it take to run similar
  routines using pure Java code? I want to make it clear, you will not find a
  large scientific benchmark run under clean room like conditions. I want to
  present a simple set of routines and how long it took to run. How long did the
  Clojure code run? How long did the Scala code run? Basically, I want to
  present the code and how long each test ran, but I don't want to claim that
  anyone language or piece of code is faster or slower based on these tests. You
  could say that most of the pure Java code ran faster. Most of the time, that
  is what happened after running these tests. But there is too much confounding
  in my tests. Like Zed Shaw said, "If you want to measure something, then don't
  measure other shit." [3] There is a lot of stuff in my tests to not make these
  an official comparison. There is a lot of confounding. But, here is the code,
  here is how long it took to run? It be relevant in more common tests like a
  Project Euler problem. Project Euler is a website that contains math problems
  intended to be solved with computer programs. In Project Euler problem number
  one, I write a program in Clojure and then in Java. They both run on the JVM
  and the same value is returned. What was the execution time for each program?
  Simple tests, simple results.

   When working with JVM languages and possible performance bottlenecks, you
   want to consider execution time, but you also want to look at the garbage
   collector and heap memory usage. Garbage collection is an expensive
   operation. It won't take a minute to run a garbage collect, but it will take
   cpu cycles away from your application. JVM code runs in a protected
   environment, the garbage collector provides automatic memory management and
   normally protects you from improper memory use. And the garbage collector
   attempts to free up memory that is no longer needed. You can't normally
   control when the JVM runs garbage collection and certainly don't want force
   it. But if you monitor your application, you can identify memory leaks or
   other problems that might cause performance bottlenecks. It will normally be
   evident where there is a problem. If you see too many garbage collects within
   a very short period of time and your total available memory is maxed out, you
   might eventually encounter an out of memory error. In a long running server
   environment, most of your performance issues might be alleviated if you look
   at proper heap memory use. Is your code forcing too many garbage collections
   within a short period of time? Are you creating too many large objects and
   holding on to them for too long? In performance tuning your application,
   there are many things to consider. It may not just be improving a particular
   algorithm. Consider heap memory and object allocation as well. For most of
   the tests, there are performance stats, memory stats and other garbage
   collection statistics.

