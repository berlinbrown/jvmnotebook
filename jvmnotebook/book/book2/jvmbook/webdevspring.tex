% %-------------------------------------- % Scala and Lift
% %--------------------------------------
\chapter{Enterprise Development with Struts2, JSF, and Spring}

With all the discussion centered around the java web-application frameworks
including Struts, SpringMVC and WebWork, how does one interface these with
Jython and why would you want to do so. Normally, you will do this the same way
that you would in a typical standalone console application. You must find a way
to invoke the Jython interpreter and then execute your Jython code. The same is
done in a Servlet environment. This example demonstrates how to put together a
web-application that uses the Struts MVC (model, view, controller) framework and
also uses Hibernate for persisting our objects to the database. The JSP files
make up all the of the View code and Jython is used for all the back-end work.
The goal of the 'BotList  Link Aggregator Application' is to create a web-app
that stores a set of links associated with keywords and description and also
presents an interface to delete, view, edit, and list the links for the user.

The Stuts Action class contains the majority of the business logic for your
web-application. In this example, the Jython classes are subclasses of the
Action class. The one Java Action class acts as a controller; depending on the
request from the user, this Action class invokes one of the Jython Action
classes accordingly. Normally, an Action will just overwrite the execute method
as shown in the Jython code below.

Our web-application would not be complete without a clear approach for
persisting the link data. So we have used the Hibernate ORM (object relational
mapping) library do the backend persistance work for us. It is not really
necessary to use Hibernate for such a simple application, but as your enterprise
application grows, the need for a more robust persistance mechanism will greatly
become evident. MySQL 5.0.2 is used for our database and most of the recent
MySQL connector APIs will work with this example.

Almost like Struts, a lot of the hibernate settings are defined in a hibernate
configuration file, 'hibernate.cfg.xml' and your hibernate mapping file,
'Botlist.hbm.xml'. Normally the most important settings for your application
include what database dialect you are using; we are using MySQL and the
definition of your hibernate POJO beans. The simple bean contains an almost
one-to-one mapping between your database fields and the Java members,
accompanied by the appropriate getters and setters.
