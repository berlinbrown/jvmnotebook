%%--------------------------------------
%% JRuby
%%--------------------------------------
\section{JRuby JVM Language}

JRuby is a JVM language.  Allows for the syntatic sugar of Ruby.

We discussed earlier how Jython is basically used for the backend 
coding, that includes communicating with Hibernate. 
Here are the code snippets associated with each of those operations. 
Most of the code is fairly intuitive; at the heart of the 
create operation, you must get the Hibernate SessionFactory 
and initiate a transaction. Once that is done, 
create an instance of the Hibernate POJO bean and populate 
the bean with the data from the Struts ActionForm. 
Once that is taken care of, use the session and transaction 
object to save the data. The Edit operation probably contains 
the most code and is seperated into two Jython classes.

% Example jruby business logic.
\begin{verbatim}
  class BotverseController
		
  def initialize(controller)
    @controller = controller
    @daohelper = @controller.entityLinksDao
  end

  # Generate the view
  def getModel(request)  
    # Audit the request
    @controller.auditLogPage(request, "botverse.html")
	query = "from org.spirit.bean.impl.BotListEntityLinks"
	postListings = @daohelper.pageEntityLinks(query, 
	         nextPage, BotListConsts::MAX_RESULTS_PAGE)
	map = BotListMapEntityLink.new    
    map['listings'] = postListings
    return map
  end

  def onSubmit(request, response, form, errors)
    link = @daohelper.readLinkListing(ratingId)    
    link.rating = link.rating + 1
    @daohelper.createLink(link)
    return form
  end

\end{verbatim}

\subsection{JVM Languages}

\subsection{Jython, Scala and Clojure}
